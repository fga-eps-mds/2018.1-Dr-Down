%% Generated by Sphinx.
\def\sphinxdocclass{report}
\documentclass[letterpaper,10pt,english]{sphinxmanual}
\ifdefined\pdfpxdimen
   \let\sphinxpxdimen\pdfpxdimen\else\newdimen\sphinxpxdimen
\fi \sphinxpxdimen=.75bp\relax

\usepackage[utf8]{inputenc}
\ifdefined\DeclareUnicodeCharacter
 \ifdefined\DeclareUnicodeCharacterAsOptional\else
  \DeclareUnicodeCharacter{00A0}{\nobreakspace}
\fi\fi
\usepackage{cmap}
\usepackage[T1]{fontenc}
\usepackage{amsmath,amssymb,amstext}
\usepackage{babel}
\usepackage{times}
\usepackage[Bjarne]{fncychap}
\usepackage{longtable}
\usepackage{sphinx}

\usepackage{geometry}
\usepackage{multirow}
\usepackage{eqparbox}

% Include hyperref last.
\usepackage{hyperref}
% Fix anchor placement for figures with captions.
\usepackage{hypcap}% it must be loaded after hyperref.
% Set up styles of URL: it should be placed after hyperref.
\urlstyle{same}

\addto\captionsenglish{\renewcommand{\figurename}{Fig.}}
\addto\captionsenglish{\renewcommand{\tablename}{Table}}
\addto\captionsenglish{\renewcommand{\literalblockname}{Listing}}

\addto\extrasenglish{\def\pageautorefname{page}}

\setcounter{tocdepth}{1}



\title{Documentacao Dr. Down}
\date{Apr 24, 2018}
\release{0.1}
\author{EPS/MDS}
\newcommand{\sphinxlogo}{}
\renewcommand{\releasename}{Release}
\makeindex

\begin{document}

\maketitle
\sphinxtableofcontents
\phantomsection\label{\detokenize{index::doc}}


Conteúdo:


\chapter{Instalando}
\label{\detokenize{install:instalando}}\label{\detokenize{install::doc}}\label{\detokenize{install:bem-vindo-a-documentacao-do-dr-down}}
\sphinxstylestrong{Observação}: Recomendamos, e o processo abaixo foi executado, uma máquina Ubuntu/Linux com a distribuição \sphinxcode{Ubuntu 16.04.4 LTS}.

O primeiro passo é fazer o clone do projeto pelo GitHub (tenha certeza de ter o \sphinxcode{git} instalado em sua máquina):

\begin{sphinxVerbatim}[commandchars=\\\{\}]
\PYGZdl{} git clone https://github.com/fga\PYGZhy{}gpp\PYGZhy{}mds/2018.1\PYGZhy{}Dr\PYGZhy{}Down.git
\end{sphinxVerbatim}

Para rodar a aplicação tenha certeza de ter algumas dependências instaladas. Existem dois scripts que auxiliam o você nessa etapa.
Para fazer a instalação basta rodar (partindo que está na pasta base após clone) os seguintes \sphinxcode{shell scripts}:

\begin{sphinxVerbatim}[commandchars=\\\{\}]
\PYGZdl{} sudo bash utility/install\PYGZus{}os\PYGZus{}dependencies.sh arg
\PYGZdl{} sudo bash utility/install\PYGZus{}python\PYGZus{}dependencies.sh
\end{sphinxVerbatim}

No primeiro script será necessário dizer qual é o \sphinxcode{arg} da operação que deseja fazer, as funções disponíveis são:
\begin{itemize}
\item {} 
list

\item {} 
help

\item {} 
install

\item {} 
upgrade

\end{itemize}

Certifique de ter instalado também:
\begin{itemize}
\item {} 
docker

\item {} 
docker-compose

\end{itemize}

E por fim, agora para rodar a aplicação basta rodar o seguinte comando no seu terminal:

\begin{sphinxVerbatim}[commandchars=\\\{\}]
\PYGZdl{} docker\PYGZhy{}compose \PYGZhy{}f local.yml up \PYGZhy{}\PYGZhy{}build
\end{sphinxVerbatim}

Com isso as imagens serão baixadas e geradas na sua máquina e você poderá acessar a aplicação pelo seu navegador no endereço \sphinxcode{127.0.0.1:8000}.

\sphinxstylestrong{Observação}: Caso deseje parar os containers basta usar a combinação \sphinxstylestrong{CTRL+C} no terminal que está rodando a aplicação, ou, caso esteja rodando em \sphinxstyleemphasis{backgroud} executar o comando:

\begin{sphinxVerbatim}[commandchars=\\\{\}]
\PYGZdl{} docker\PYGZhy{}compose \PYGZhy{}f local.yml down
\end{sphinxVerbatim}


\section{Configurando seu usuário:}
\label{\detokenize{install:configurando-seu-usuario}}\begin{itemize}
\item {} 
Para criar um \sphinxstylestrong{conta de usuário normal}, vá em Criar Conta e preencha os campos. Assim que você submeter suas informações, você verá uma página de \sphinxquotedblleft{}Verificar seu endereço de E-mail\sphinxquotedblright{}. Vá no seu terminal, no seu console você verá uma mensagem de email de verificação. Copie o link para seu negador. Agora o E-mail do usuário deve ser verificado e pronto para ser usado.

\item {} 
Para criar uma \sphinxstylestrong{conta super usuário}, use esse comando:

\begin{sphinxVerbatim}[commandchars=\\\{\}]
\PYGZdl{} docker\PYGZhy{}compose \PYGZhy{}f local.yml run \PYGZhy{}\PYGZhy{}rm django python manage.py createsuperuser
\end{sphinxVerbatim}

\end{itemize}

Por conveniência, você pode manter o seu usuário normal logado no Chrome e seu super usuário (administrador) logado no Firefox (ou similar), assim você consegue ver como o site se comporta em ambos usuários.

\sphinxstylestrong{Observação}: O tutorial acima mostra como instalar e rodar a máquina em ambiente de desenvolvimento, para ambiente de produção, verifique o deploy.


\chapter{Deploy}
\label{\detokenize{deploy::doc}}\label{\detokenize{deploy:deploy}}

\section{Homologação (\sphinxstyleemphasis{staging})}
\label{\detokenize{deploy:homologacao-staging}}
O deploy em ambiente de testes foi feito com o auxílio de algumas documentações, principalmente a que pode ser acessada pelo projeto do cookiecutter, na seção \sphinxhref{https://cookiecutter-django.readthedocs.io/en/latest/deployment-with-docker.html}{Deploy com Docker}, atentado as particularidades, dado ao fato desse ser o ambiente de homologação da ferramenta.

Para o deploy é necessário:
\begin{itemize}
\item {} 
Um droplet do \sphinxhref{https://www.digitalocean.com/}{Digital Ocean} (Ubuntu 16.04.4 LTS);

\item {} 
Um domínio (foi usado o provedor \sphinxhref{http://www.freenom.com/pt/index.html}{Freenom});

\end{itemize}

\sphinxstylestrong{Observação}: A equipe decidiu por usar as máquinas (droplets) do \sphinxstyleemphasis{Digital Ocean} para o deploy, mas uma máquina com um IP público já é apropriada para o processo.


\subsection{Configurando o domínio}
\label{\detokenize{deploy:configurando-o-dominio}}
Com uma máquina em mãos o primeiro passo é configurar o seu domínio para que o nome o qual você registrou possa apontar para a máquina (IP) a qual você irá fazer o deploy da aplicação.
Neste passo foi seguida a documentação do \sphinxstyleemphasis{Digital Ocean}, que apresenta como configurar o \sphinxhref{https://www.digitalocean.com/community/tutorials/an-introduction-to-digitalocean-dns}{DigitalOcean DNS}.
Em poucos passos o que foi feito:
\begin{itemize}
\item {} 
Adicionado o domínio registrado no \sphinxstyleemphasis{Digital Ocean};

\item {} 
Adicionado os registros do DNS;

\item {} 
E configurado os \sphinxstyleemphasis{Nameservers} no site do registro do domínio.

\end{itemize}

Com isso sua máquina já estará mapeada no domínio registrado e já será possível acessá-la via URL do domínio.


\subsection{Fazendo o deploy na máquina}
\label{\detokenize{deploy:fazendo-o-deploy-na-maquina}}
Para fazer o deploy na sua máquina, que agora já tem o IP registrado num domínio, se atente em ter as seguintes dependências instaladas:
\begin{itemize}
\item {} 
Nginx

\item {} 
Docker

\item {} 
Docker Compose

\end{itemize}

Acesse sua máquina via SSH e faça os \sphinxhref{https://github.com/fga-gpp-mds/2018.1-Dr-Down/blob/develop/docs/install.rst}{passos de instalação}.

Agora é necessário configurar o Nginx para que ele possa mapear as portas e servir os arquivos da sua instalação.
Aconselhamos \sphinxhref{https://linode.com/docs/web-servers/nginx/how-to-configure-nginx/}{este} tutorial ou se preferir \sphinxhref{https://www.nginx.com/resources/wiki/start/topics/examples/full/}{esta} documentação, que é o exemplo encontrado no site oficial do Nginx.
Mas é importante ressaltar que ao instalar o \sphinxstyleemphasis{nginx} na sua máquina, o mesmo cria um arquivo \sphinxstyleemphasis{default} de configuração e a ferramenta do Dr. Down já trabalha automaticamente com essa porta, logo com a configuração \sphinxstyleemphasis{default} a aplicação deve rodar sem problema, caso precise mudar ou especificar algo diferente, leia os documentos supracitados.
Com isto certifique que:
\begin{itemize}
\item {} 
Tem as dependências instaladas;

\item {} 
As instruções de instalação do software foram seguidas;

\item {} 
Os \sphinxstyleemphasis{containers} docker estão rodando em máquina (status \sphinxstyleemphasis{Up}).

\end{itemize}

Caso todos os passos acima tenham sido feitos, agora, basta acessar a URL do seu domínio que o site Dr. Down deverá estar disponível para o ambiente de homologação.

\sphinxstylestrong{Observação}: Caso enfrente algum problema, sempre verique os logs de máquina, e não hesite em nos contatar via \sphinxhref{https://github.com/fga-gpp-mds/2018.1-Dr-Down/issues/new}{Issue} no GitHub, faremos o possível para ajudar.

\sphinxstylestrong{Observação}: Os comandos de máquina como \sphinxstyleemphasis{makemigrations}, \sphinxstyleemphasis{migrate}, \sphinxstyleemphasis{shell}, \sphinxstyleemphasis{createsuperuser} e etc, devem ser feitos na máquina host da aplicação, assim como a verficação de usuários por e-mail deverá ser feita como o padrão em desenvolvimento (pegar o link de verificação gerado pelo \sphinxstyleemphasis{output} do console da aplicação).


\subsection{Deploy Contínuo}
\label{\detokenize{deploy:deploy-continuo}}
A aplicação Dr. Down tem um pipeline de deploy contínuo para o ambiente de homologação, ele é executado junto com os testes e a \sphinxstyleemphasis{build} nos \sphinxstyleemphasis{jobs} do \sphinxhref{https://travis-ci.org/}{Travis CI}, caso queira manter o mesmo pipeline usado por nós modifique o arquivo \sphinxcode{.travis.yml} que fica na pasta raíz do projeto, se atentando em configurar de acordo com as suas necessidades os seguintes parâmetros:
\begin{itemize}
\item {} 
Usuário do Docker Hub;

\item {} 
Senha de acesso ao Docker Hub;

\item {} 
IP da máquina de deploy;

\item {} 
Senha de acesso a máquina de deploy.

\end{itemize}

Com esses parâmetros devidamente configurados, o deploy contínuo deve funcionar normalmente no seu contexto.


\section{Produção (\sphinxstyleemphasis{production})}
\label{\detokenize{deploy:producao-production}}
O deploy em ambiente de produção ainda está em fase de teste, portanto ainda não será documentado aqui.


\chapter{Testes}
\label{\detokenize{tests::doc}}\label{\detokenize{tests:testes}}
Nesta seção iremos demonstrar alguns processos de teste da aplicação Dr. Down, além de mostrar como você pode escrever os seus próprios testes.
A aplicação Dr. Down vem com duas suítes de teste python instaladas e configuradas, sendo uma o py.test e a outra o coverage.
Antes de explicar por menor as suítes de teste vamos ver como são feitos os testes em python/Django.


\section{Testes python/Django}
\label{\detokenize{tests:testes-python-django}}
O Django provê ao desenvolvedor diversas formas de testar sua aplicação, diversas ferramentas \sphinxstyleemphasis{third-party} também estão disponíveis para auxiliar na tarefa dos testes.
Testar uma aplicação web é uma tarefa complexa, porque uma aplicação web é feita por diversas camadas lógicas \textendash{} vai do nível de uma requisição HTTP, a uma validação e processamento de um formulário, até a renderização de um template. Com a execução de teste em Django e dispondo de diversas utilidades, você pode simular requisições, inserir informações testes, inspecionar saídas de sua aplicação e geralmente verificar se seu código está fazendo o que ele deveria fazer.
Por padrão os testes Django devem estar na pasta \sphinxcode{tests/} do seu app, com isso as ferramentas reconhecem como teste do seu código e são executadas.
Algumas documentações de referência são:


\subsection{Escrevendo e Executando Testes}
\label{\detokenize{tests:testes-unitarios}}\label{\detokenize{tests:id1}}

\subsection{Testes Unitários}
\label{\detokenize{tests:id2}}

\section{Testando Dr. Down}
\label{\detokenize{tests:testando-dr-down}}

\subsection{py.test:}
\label{\detokenize{tests:py-test}}
Para apenas rodar a suíte de testes com o py.test basta executar o seguinte comando:

\begin{sphinxVerbatim}[commandchars=\\\{\}]
\PYGZdl{} docker\PYGZhy{}compose \PYGZhy{}f local.yml run \PYGZhy{}\PYGZhy{}rm django py.test
\end{sphinxVerbatim}

\sphinxstylestrong{Observação}: No próprio terminal será mostrado o \sphinxstyleemphasis{output} dos testes rodados. Outras configurações, \sphinxstyleemphasis{flags} e modos de uso do py.test podem ser verificadas na documentação do \sphinxhref{https://docs.pytest.org/en/latest/contents.html}{py.test}.


\subsection{coverage:}
\label{\detokenize{tests:coverage}}
O Coverage é uma ferramenta auxiliar que permite gerar relatórios e verifcar a cobertura de código a partir dos testes executados. Tanto que para isso ele usa o auxílio do py.test, no entanto, há forma de executar apenas o coverage, ou até usar de outros \sphinxstyleemphasis{frameworks} de teste, mas nesse projeto utilizamos ele em conjunto com o py.test, sua execução se dá da seguinte forma:

\begin{sphinxVerbatim}[commandchars=\\\{\}]
\PYGZdl{} docker\PYGZhy{}compose \PYGZhy{}f local.yml run \PYGZhy{}\PYGZhy{}rm django coverage run \PYGZhy{}m py.test
\PYGZdl{} docker\PYGZhy{}compose \PYGZhy{}f local.yml run \PYGZhy{}\PYGZhy{}rm django coverage html
\PYGZdl{} firefox htmlcov/index.html
\end{sphinxVerbatim}

Com estes comandos o coverage irá executar o testes com auxílio do py.test, vai utilizar os dados dele como input para gerar a cobertura de testes e mostrar ao usuário.

\sphinxstylestrong{Observação}: Será gerada uma pasta \sphinxcode{htmlcov/} que conterá o \sphinxcode{index.html}, este arquivo contem o relatório extraído dos testes rodados, no exemplo acima foi utilizado o navegador \sphinxstyleemphasis{Mozilla Firefox} para a abertura do arquivo HTML gerado. Outras configurações, \sphinxstyleemphasis{flags} e modos de uso do coverage podem ser verificadas na documentação do \sphinxhref{https://coverage.readthedocs.io/en/coverage-4.5.1/}{coverage}.

Dúvidas ou problemas, acesse nosso \sphinxhref{https://github.com/fga-gpp-mds/2018.1-Dr-Down}{github} e fique a vontade para abrir uma \sphinxstyleemphasis{issue} para que possamos esclarecer!


\chapter{Indíces e Tabelas}
\label{\detokenize{index:indices-e-tabelas}}\begin{itemize}
\item {} 
\DUrole{xref,std,std-ref}{genindex}

\item {} 
\DUrole{xref,std,std-ref}{modindex}

\item {} 
\DUrole{xref,std,std-ref}{search}

\end{itemize}



\renewcommand{\indexname}{Index}
\printindex
\end{document}